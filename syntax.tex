\section{The Language DIF}
Figure \ref{figure_grammar} gives the syntax of DIF. The syntax is similar to
that of DOT in \cite{AGORS16}, with the addition of constructs related to
implicit functions. We add a new type $\IFUN{u}{S}{S'}$ which represents an
implicit (path-dependent) function from $S$ to $S'$. We also add the
\emph{implicit query} $\IMP$, the analogue to Scala's \texttt{implicitly[T]}.
Occurrences of $\IMP$ are \emph{resolved} into variables by our typing and
translation rules. The variable that replaces a given occurrence of $\IMP$ is
chosen from candidate implicit variables, which are bound by let constructs,
where $\IMP$ is the variable name, i.e. $\LET{\IMP}{t}{t'}$.

Note that we make a distinction between two sets of variable names. The first,
ranged over by $x, y$ is the set of variable names with the implicit query. The
second set, ranged over by $u, v$ is the set of variables where $\IMP$ is not
allowed. At the term level, anywhere we can write a variable name, we could
also write an implicit query, so we use $x$ in the grammar of terms. At the
type level, however, implicit queries are not allowed in names, and we
therefore use $u$ for names in the grammar of types.

\begin{figure*}[h]
    \begin{minipage}[t]{.5\textwidth}\[\begin{array}{r c l l}
    \multicolumn{4}{l}{\textsc{Terms}}                        \\
    t & ::= & x.a             & \textit{Selection           } \\
      &  |  & x               & \textit{Variable            } \\
      &  |  & x\ y            & \textit{Application         } \\
      &  |  & \LET{x}{t}{t'}  & \textit{Let-binding         } \\
      &  |  & \OBJ{x}{S}{d}   & \textit{Object              } \\
      &  |  & \ABS{x}{S}{t}   & \textit{Abstraction         } \\[3mm]
    \multicolumn{4}{l}{\textsc{Types}}                        \\
    S & ::= & \FDEC{a}{S}     & \textit{Field Declaration   } \\
      &  |  & \AGG{S}{S'}     & \textit{Intersection        } \\
      &  |  & \REC{u}{S}      & \textit{Recursive type      } \\
      &  |  & \top            & \textit{Top                 } \\
      &  |  & \bot            & \textit{Bottom              } \\
      &  |  & \TDEC{A}{S}{S'} & \textit{Field Declaration   } \\
      &  |  & u.A             & \textit{Type Projection     } \\
      &  |  & \DFUN{u}{S}{S'} & \textit{Dependent Function  } \\[3mm]
    \multicolumn{4}{l}{\textsc{Values}} \\
    v & ::= & \multicolumn{2}{l}{\OBJ{x}{S}{d}
        \quad | \quad \ABS{x}{S}{t}} \\[3mm]
\end{array}\]\end{minipage}
\begin{minipage}[t]{.4\textwidth}\[\begin{array}{r c l l}
    \multicolumn{4}{l}{\textsc{Definitions}}                  \\
    d & ::= & \DEF{A}{S}      & \textit{Type Definition     } \\
      &  |  & \DEF{a}{t}      & \textit{Field Definition    } \\
      &  |  & \AGG{d}{d'}     & \textit{Aggregate Definition} \\[3mm]
    \multicolumn{4}{l}{\textsc{Contexts}} \\
    e & ::= & \multicolumn{2}{l}{\HOL
        \quad | \quad \LET{x}{\HOL}{t}
        \quad | \quad \LET{x}{v}{e}} \\[3mm]
    \multicolumn{4}{l}{\textsc{Variables (with implicit query)}} \\
    \multicolumn{4}{l}{x, y\ ::=\ p \quad | \quad i \quad | \quad \IMP} \\[3mm]
    \multicolumn{4}{l}{\textsc{Variables}} \\
    \multicolumn{4}{l}{u, v\ ::=\  p \quad | \quad i} \\[3mm]
    \multicolumn{4}{l}{\textsc{Explicit Variables} \quad p, q, ...} \\[3mm]
    \multicolumn{4}{l}{\textsc{Implicit Variables} \quad i, j, ...} \\[3mm]
    \multicolumn{4}{l}{\textsc{Implicit Query    } \quad \IMP     } \\[3mm]
    \multicolumn{4}{l}{\textsc{Term Members      } \quad a, b, ...} \\[3mm]
    \multicolumn{4}{l}{\textsc{Type Members      } \quad A, B, ...} \\[3mm]
\end{array}\]\end{minipage}

    \caption{Grammar of DIF}
    \label{figure_grammar}
\end{figure*}

\subsection{Abbreviations}
We employ the following abbreviations in examples:

\noindent -- We encode multiple argument function (types) as multiple single
function (types):
\begin{align*}
    \forall(y: T, z: U) &\equiv \forall(y: T)\forall(z: U) \\
    \lambda(y: T, z: U) &\equiv \lambda(y: T)\lambda(z: U)
\end{align*}
\noindent -- We group intersection contents together inside $\{$ curly brackets
$\}$, separating definitions with a newline, additional whitespace or semicolon:
\begin{align*}
    \{d \quad d'\} &\equiv \{d\} \wedge \{d'\} \\
    \{d \ ;\  d'\} &\equiv \{d\} \wedge \{d'\}
\end{align*}
\noindent -- We allow terms in application and selection by encoding them as
let-bindings:
\begin{align*}
    t\ t' &\equiv \LET{x}{t}{x\ t'} \\
    x\ t &\equiv \LET{y}{t}{x\ y} \\
    t.a &\equiv \LET{x}{t}{x.a}
\end{align*}
\noindent -- We abbreviate type bounds thusly:
\begin{align*}
    A <: T &\equiv A: \bot..T & A = T &\equiv A: T..T \\
    A >: T &\equiv A: T..\top & A &\equiv A: \bot..\top
\end{align*}
\noindent -- We encode type ascription as application:
\begin{align*}
    t: T &\equiv (\ABS{x}{T}{x})t \\
    \LET{x: T}{t}{t'} &\equiv \LET{x}{t: T}{t'}
\end{align*}
\noindent -- Finally we omit types in new-bindings if the type of the definition
is explicit, writing $\OBJ{x}{T}{d}$ as $\nu(x)d$.

\subsection{Semantics}
We define the semantics of DIF by type-based translation from DIF programs to
DOT programs. The meaning of a DIF program is therefore given by its typed
translation into a DOT program. This translation is the topic of section
\ref{typing}.
