\section{Typing for DIF}
\label{typing}

In this section we introduce the typing system for DIF programs. Typing
judgements in DIF are 4-place: $\TJ{\Gamma}{t}{S}{\HAT{t}}$. This judgement can
be read: under environment $\Gamma$, the DIF term $t$ has type $S$, and
translated to the DOT term $\HAT{t}$.

\subsection{Translation of Types}

We define the function $\TRANS{(\bullet)}$, which translates DIF types into DOT types. DIF types differ from DOT types only in the inclusion of the type for implicit functions $\IFUN{u}{S}{S'}$. The translation simply erases occurrences of $\IMP$, translating implicit functions into explicit functions.

\begin{DEFINITION}[Translation of types]
    \label{translation}
    Figure \ref{figure_translation} defines translation from DIF types to DOT
    types.
\end{DEFINITION}

\begin{figure}[h]
    \begin{align*}
    \TRANS{a} \                     &= \ a
        & \TRANS{\REC{u}{S}}        &= \REC{u}{\TRANS{S}}
        & \TRANS{\TDEC{A}{S}{S'}}   &= \TDEC{A}{\TRANS{S}}{\TRANS{S'}} \\
    \TRANS{\FDEC{a}{S}}             &= \FDEC{a}{\TRANS{S}}
        & \TRANS{\top}              &= \top
        & \TRANS{(u.A)}             &= u.\TRANS{A}                     \\
    \TRANS{(\AGG{S}{S'})}           &= \AGG{\TRANS{S}}{\TRANS{S'}}
        & \TRANS{\bot}              &= \bot
        & \TRANS{(\DFUN{u}{S}{S'})} &= \DFUN{u}{\TRANS{S}}{\TRANS{S'}} \\
    & & & & \TRANS{(\IFUN{u}{S}{S'})} &= \DFUN{u}{\TRANS{S}}{\TRANS{S'}} \\
\end{align*}

    \caption{Translation from DIF types to DOT types}
    \label{figure_translation}
\end{figure}

We extend the definition of $\TRANS{(\bullet)}$ pointwise to environments
$\Gamma$.

\subsection{Type substitution}

\begin{DEFINITION}[Name substitution on types]
    \label{substitution}
    We define capture-avoiding substitution of names within types, written
    $\SUB{u}{v}S$, in the usual way.
\end{DEFINITION}

\subsection{The functions \textit{depth} and \textit{spec}}

The function $\DEPTH{\Gamma, i, j}$ decides which variable is more deeply nested
in the environment $\Gamma$, for purposes of disambiguation of implicit variable
selection. It returns $-1$ if $i$ is more deeply nested, and $1$ if $j$ is more
deeply nested. Its formal definition is given in figure \ref{figure_depth}.

\begin{figure}[h]
    \[\begin{array}{c}
    \DEPTH{\Gamma, i, j} =
        \begin{cases}
            -1   & \IF \Gamma = \Gamma_1, i: S_1, \Gamma_2, j: S_2, \Gamma_3
                \wedge \{i, j\} \cap \DOM{\Gamma_1 \cup \Gamma_2 \cup \Gamma_3}
                \neq \o \\
            1    & \IF \Gamma = \Gamma_1, j: S_1, \Gamma_2, i: S_2, \Gamma_3
                \wedge \{i, j\} \cap \DOM{\Gamma_1 \cup \Gamma_2 \cup \Gamma_3}
                \neq \o \\
            \bot & \ \textbf{otherwise}
        \end{cases}
\end{array}\]

    \caption{The \textbf{\textit{depth}} function}
    \label{figure_depth}
\end{figure}

The function $\SPEC{\Gamma, i, j}$ decides which variable's type is more
specific in the environment $\Gamma$, i.e. which type can be instantiated to the
other by widening or polymorphic parameter instantiation. It returns $-1$ if $i$
is more specific, $1$ if $j$ is more specific, and if they are equal (i.e. both
a subtype and a supertype of each other), 0 is returned. Its formal definition
is given in figure \ref{figure_spec}.

\begin{figure}[h]
    \[\begin{array}{c}
    \SPEC{\Gamma, i, j} = \begin{cases}
        -1   & \IF \SJ{\Gamma}{i}{j} \wedge \neg(\SJ{\Gamma}{j}{i}) \\
        1    & \IF \neg(\SJ{\Gamma}{i}{j}) \wedge \SJ{\Gamma}{j}{i} \\
        0    & \IF \SJ{\Gamma}{i}{j} \wedge \SJ{\Gamma}{j}{i} \\
        \bot & \ \textbf{otherwise}
    \end{cases}
\end{array}\]

    \caption{The \textbf{\textit{spec}} function}
    \label{figure_spec}
\end{figure}

\subsection{Typing rules}

Figure \ref{figure_binding_rules} gives the binding rules for DIF binders.
Binding rules are two-place judgements of the form $x \leadsto u$. Every time a
binder is encountered in the typing system, a binding rule will decide if that
binder should be replaced with a fresh variable name (in the case that the
binder is the implicit query $\IMP$) or left unchanged. We allow use of $\IMP$
as a binder in several places, for example in binding an implicit variable:
$\LET{\IMP}{...}{...}$. The rule \RULE{Bind-Im} generates a fresh name in such a
case, and occurrences of the implicit query to be resolved to the variable at
the binder are replaced with the variable generated by the binding rule. The
rule \RULE{Bind-Ex} leaves explicit variable bindings unchanged.

\begin{figure}[h]
    \[\begin{array}{c}
    \RULE{Bind-Ex} \quad p \leadsto p \qquad
    \RULE{Bind-Im} \quad \inference{
        i\ \text{fresh}
    }{
        \IMP \leadsto i
    }
\end{array}\]

    \caption{Binding rules for DIF terms}
    \label{figure_binding_rules}
\end{figure}

Figure \ref{figure_typing_rules_terms} gives the binding rules for DIF terms.
Most rules are identical to their corresponding rules in DOT (specifically in
\cite{AGORS16}), when ignoring the translation part of typing judgements
($\leadsto \HAT{t}$). The typing rules differ slightly from their DOT
counterparts in that they have binding judgements in their premises. The rules
\RULE{Var-Ex}, \RULE{All-Ex-I} and \RULE{All-Ex-E} are analogous to the rules
\RULET{Var}, \RULET{All-I} and \RULET{All-E} respectively. The rule
\RULE{All-Im-I} types introduction of implicit functions, introducing an
(explicit) abstraction to the translation. The rule \RULE{All-Im-E} types
implicit function elimination, inserting an implicit variable of suitable type
as an argument to the implicit function. Finally \RULE{Var-Im} types implicit
query, replacing $\IMP$ with a chosen implicit variable in the translation.
This rule also performs \emph{disambiguation}, whenever more than one implicit
variable is of suitable type. Disambiguation follows the same process as Scala
(as specified in \cite{OBLB18}). We prefer more deeply nested implicit
variables over less deeply nested ones, and implicit variables with more
specific types over ones with more general types. Where we have a choice
between two variables, one of which is more deeply nested and the other a more
specific type, we reject the program as ambiguous.

\begin{figure*}[h]
    \[\begin{array}{c}
    \RULE{Var-Ex} \quad \TJ{\Gamma, p: S, \Gamma'}{p}{S}{p} \qquad

    \RULE{Var-Im} \quad \TJ{\Gamma, i: S, \Gamma'}{\IMP}{S}{i} \\[4mm]

    \RULE{Sub} \quad \inference{
        \TJ{\Gamma}{t}{S}{\HAT{t}} \quad
        \SJT{\Gamma}{S}{S'}
    }{
        \TJ{\Gamma}{t}{S'}{\HAT{t}}
    } \\[4mm]

    \RULE{All-I} \quad \inference{
        x \leadsto u \quad
        \TJ{\Gamma, u: S}{t}{U}{\HAT{t}} \quad
        \{x, u\} \cap \FV{S} = \o
    }{
        \TJ{\Gamma}{\ABS{x}{S}{t}}{\DFUN{u}{S}{U}}{\ABS{u}{S}{\HAT{t}}}
    } \qquad

    \RULE{All-E} \quad \inference{
        \TJ{\Gamma}{x}{\DFUN{u}{S}{U}}{\HAT{x}} \quad
        \TJ{\Gamma}{y}{S}{\HAT{y}}
    }{
        \TJ{\Gamma}{x\ y}{\SUB{u}{y}U}{\HAT{x}\ \HAT{y}}
    } \\[4mm]

    \RULE{Let} \quad \inference{
        x \leadsto u \quad
        \TJ{\Gamma}{t}{S}{\HAT{t}} \quad
        \TJ{\Gamma, u: S}{t'}{U}{\HAT{t'}} \quad
        x \notin \FV{U}
    }{
        \TJ{\Gamma}{\LET{x}{t}{t'}}{U}{\LET{u}{\HAT{t}}{\HAT{t'}}}
    } \qquad

    \RULE{And-I} \quad \inference{
        \TJ{\Gamma}{x}{S}{\HAT{x}} \quad
        \TJ{\Gamma}{x}{U}{\HAT{x}}
    }{
        \TJ{\Gamma}{x}{\AGG{S}{U}}{\HAT{x}}
    } \\[4mm]

    \RULE{\{\}-I} \quad \inference{
        x \leadsto u \quad
        \TJ{\Gamma, u: S}{d}{S}{\HAT{d}}
    }{
        \TJ{\Gamma}{\OBJ{x}{S}{d}}{\REC{u}{S}}{\OBJ{u}{S}{\HAT{d}}}
    } \qquad

    \RULE{Fld-E} \quad \inference{
        \TJ{\Gamma}{x}{\FDEC{a}{S}}{\HAT{x}}
    }{
        \TJ{\Gamma}{x.a}{S}{\HAT{x}.a}
    } \\[4mm]

    \RULE{Rec-I} \quad \inference{
        \TJ{\Gamma}{x}{S}{\HAT{x}}
    }{
        \TJ{\Gamma}{x}{\REC{x}{S}}{\HAT{x}}
    } \qquad

    \RULE{Rec-E} \quad \inference{
        \TJ{\Gamma}{x}{\REC{x}{S}}{\HAT{x}}
    }{
        \TJ{\Gamma}{x}{S}{\HAT{x}}
    }
\end{array}\]

    \caption{Typing and translation rules for DIF terms}
    \label{figure_typing_rules_terms}
\end{figure*}

Figure \ref{figure_typing_rules_definitions} gives the typing rules for DIF
definitions. These are again very similar to the typing rules for DOT
defintions, and in each case the translations are homomorphic.

\begin{figure*}[h]
    \[\begin{array}{c}
    \RULE{Typ-I} \quad \TJT{\Gamma}{\DEF{A}{S}}{\TDEC{A}{S}{S}} \qquad

    \RULE{Fld-I} \quad \inference{
        \TJT{\Gamma}{t}{S}
    }{
        \TJT{\Gamma}{\DEF{a}{t}}{\FDEC{a}{S}}
    } \\[4mm]

    \RULE{AndDef-I} \quad \inference{
        \TJT{\Gamma}{d_1}{S_1} \quad
        \TJT{\Gamma}{d_2}{S_2} \quad
        \DOM{d_1} \cap \DOM{d_2} = \o
    }{
        \TJT{\Gamma}{\AGG{d_1}{d_2}}{\AGG{S_1}{S_2}}
    }
\end{array}\]

    \caption{Typing and translation rules for DIF definitions}
    \label{figure_typing_rules_definitions}
\end{figure*}

Figure \ref{figure_subtyping_rules} gives the subtyping rules for DIF terms and
definitions. These are yet again very similar to the subtyping rules in DOT.
The exception is the rule \RULE{All-<:-All-Im}, which makes explicit functions
a subtype of implicit functions. This allows the passing of arguments
explicitly to an implicit function, which is an important aspect of implicit
functions in Scala, as it allows overriding as necessary any implicit variable
the compiler would otherwise insert.

\begin{figure*}[h]
    \[\begin{array}{c}
    \RULE{Top} \quad \SJ{\Gamma}{S}{\top} \qquad

    \RULE{Bot} \quad \SJ{\Gamma}{\bot}{S} \qquad

    \RULE{Refl} \quad \SJ{\Gamma}{S}{S} \\[4mm]

    \RULE{Trans} \quad \inference{
        \SJ{\Gamma}{S}{U} \quad
        \SJ{\Gamma}{U}{R}
    }{
        \SJ{\Gamma}{S}{R}
    } \qquad

    \RULE{<:-And} \quad \inference{
        \SJ{\Gamma}{S}{U} \quad
        \SJ{\Gamma}{S}{R}
    }{
        \SJ{\Gamma}{S}{\AGG{U}{R}}
    } \\[4mm]

    \RULE{<:-Sel} \quad \inference{
        \TJ{\Gamma}{u}{\TDEC{A}{S}{U}}{u}
    }{
        \SJ{\Gamma}{S}{u.A}
    } \qquad

    \RULE{Sel-<:} \quad \inference{
        \TJ{\Gamma}{u}{\TDEC{A}{S}{U}}{u}
    }{
        \SJ{\Gamma}{u.A}{U}
    } \\[4mm]

    \RULE{All-<:-All-Ex} \quad \inference{
        \SJ{\Gamma}{S_2}{S_1} \quad
        \SJ{\Gamma, x: S_2}{U_1}{U_2}
    }{
        \SJ{\Gamma}{\DFUN{x}{S_1}{U_1}}{\DFUN{x}{S_2}{U_2}}
    } \\[4mm]

    \RULE{Typ-<:-Typ} \quad \inference{
        \SJ{\Gamma}{S_2}{S_1} \quad
        \SJ{\Gamma}{U_1}{U_2} \quad
    }{
        \SJ{\Gamma}{\TDEC{A}{S_1}{U_1}}{\TDEC{A}{S_2}{U_2}}
    } \\[4mm]

    \RULE{And$_1$-<:} \quad \SJ{\Gamma}{\AGG{S}{U}}{S} \qquad

    \RULE{And$_2$-<:} \quad \SJ{\Gamma}{\AGG{S}{U}}{U} \\[4mm]

    \RULE{Fld-<:-Fld} \quad \inference{
        \SJ{\Gamma}{S}{U}
    }{
        \SJ{\Gamma}{\FDEC{a}{S}}{\FDEC{a}{U}}
    } \\[4mm]

    \RULE{All-<:-All-Im} \quad \inference{
        \SJ{\Gamma}{S_2}{S_1} \quad
        \SJ{\Gamma, x: S_2}{U_1}{U_2}
    }{
        \SJ{\Gamma}{\IFUN{x}{S_1}{U_1}}{\IFUN{x}{S_2}{U_2}}
    }
\end{array}\]

    \caption{Subtyping rules for DIF}
    \label{figure_subtyping_rules}
\end{figure*}
