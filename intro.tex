\section{Introduction}

\subsection{Implicit functions}
Modularity, a core concept in software engineering, is greatly aided by
parameterisation of programs. Parameterisation has dual facets: supplying and
consuming a parameter.  A key tension in large-scale software engineering is
between \emph{explicit} (e.g.~pure functional programming), and \emph{implicit}
parameterisation (e.g.~global state).  The former enables local reasoning but
can lead to repetitive supply of parameters.  Here is a simple example of the
problem:

\begin{minipage}{\linewidth}
\begin{lstlisting}[mathescape]
def compare(x: Int, y: Int)(comparator:
    Int => Int => Boolean): Boolean =
  comparator(x)(y)
...
compare(3, 4)(<=)
...
compare(17, 12)(<=)
...
\end{lstlisting}
\end{minipage}

\noindent Repeatedly passing functions like \texttt{<=} which are unlikely to
change frequently, is tedious, and impedes readability of large code bases.
Default parameters  are an early proposal for mediating this tension in a
type-safe way. The key idea is to annotate function arguments with their
default value, to be used whenever an invocation does not supply an argument:

\begin{minipage}{\linewidth}
\begin{lstlisting}[mathescape]
def compare(x: Int, y: Int)(comparator:
    Int => Int => Boolean = <=): Boolean =
  comparator(x)(y)
...
compare(3, 4)
...
compare(17, 12)
...
\end{lstlisting}
\end{minipage}

\noindent The compiler synthesises $\texttt{compare(3, 4)(<=)}$ from
$\texttt{compare(3, 4)}$, and $\texttt{compare(17, 12)(<=)}$ from
$\texttt{compare(17, 12)}$. The missing argument indicates to the compiler that
the default $\texttt{<=}$ should be used. Default parameters have a key
disadvantage: the default value is hard-coded at the callee, and cannot be
context dependent. Implicit arguments, a strict generalisation of default
parameters, were pioneered in Haskell \cite{LLMS00}, and popularised as well as
refined  in Scala \cite{OBLB18}: they separate the \emph{callee's declaration}
that an argument can be elided, from the \emph{caller's choice} of elided
values, allowing the latter to be context dependent.

\begin{minipage}{\linewidth}
\begin{lstlisting}[mathescape]
def compare(x: Int, y: Int)(implicit comparator:
    Int => Int => Boolean): Boolean =
  comparator(x)(y)
...
implicit val cmp = <=
compare(3, 4)
...
implicit val cmp = >
compare(17, 12)
...
\end{lstlisting}
\end{minipage}

\noindent In this example $\texttt{compare(3, 4)}$ is rewritten as above, but
$\texttt{compare(17, 12)}$ becomes $\texttt{compare(17, 12)(>)}$, i.e.~a
different implicit argument is synthesised. The disambiguation between several
providers of implicit arguments happens at compile-time using type and scope
information. Programs where elided arguments cannot be disambiguated at
compile-time are rejected as ill-formed. Hence type-safety is not compromised.

\subsection{Type classes}
Type classes \cite{K88, WB89} are a mechanism to allow ad-hoc polymorphism in
languages with parametric polymorphism. Parametrically polymorphic type
variables may be \emph{constrained}, such that they can only be instantiated by
types over which a user-defined set of functions exist. Further functions can
then be defined over all types for which the set of functions are defined. A
typical example of a type class is \texttt{Show}. Note that for this example we
use Haskell-style listings, as Haskell is the most well-known language with type
classes. In a single program, we might want to use a function \texttt{print} to
print values over many different types, for example we might want to write
\texttt{print 1}, \texttt{print True} and \texttt{print [1, 2]} in the same
program. To acheive this with type classes, we give \texttt{print} the following
signature:
\begin{lstlisting}[mathescape]
    print :: Show a => a -> IO ()
\end{lstlisting}

\noindent This signature tells us that where \texttt{a} is a type in the
\texttt{Show} type class, \texttt{print} takes an \texttt{a} and returns
\texttt{IO ()} (in Haskell, \texttt{()} is the unit type and the type
constructor \texttt{IO} represents a type obtained as a result of performing
IO). The definition of the \texttt{Show} type class itself is given below:
\begin{lstlisting}[mathescape]
    class Show a where
        show :: a -> String
\end{lstlisting}

\noindent We call this a \emph{class definition}. It tells us that a type
\texttt{a} is in the \texttt{Show} type class when there exists a function
\texttt{show} of type \texttt{a -> String}. We can then define a function for a
given type, for example \texttt{Bool}, as below, in an \emph{instance
definition}:
\begin{lstlisting}[mathescape]
    instance Show Bool where
        show b = if b then "True" else "False"
\end{lstlisting}

\noindent And for \texttt{Int}, assuming a function \texttt{intToString :: Int
-> String} that converts an \texttt{Int} to its \texttt{String} representation:
\begin{lstlisting}[mathescape]
    instance Show Int where
        show i = intToString i
\end{lstlisting}

\noindent We can also declare type constructors to be in a type class when their
argument types are also in the type class. For example, we can declare that
tuples of type \texttt{(a, b)} for all \texttt{a}, \texttt{b} are in
\texttt{Show} when \texttt{a} and \texttt{b} are in \texttt{Show} thusly:
\begin{lstlisting}[mathescape]
    instance (Show a, Show b) => Show (a, b) where
        show (a, b) = "(" ++ show a ++ ", " ++
                                 show b ++ ")"
\end{lstlisting}

\noindent With these instance definitions at hand, we can give a definition for
\texttt{print}, which will satisfy the type checker for arguments of any type in
the \texttt{Show} type class, as above. The definition is as follows:
\begin{lstlisting}[mathescape]
    print a = putStrLn (show a)
\end{lstlisting}

Type classes are implemented by via a technique known as \emph{dictionary
passing}. Function definitions whose type contains a type class constrained type
variable are adapted to take an additional parameter for each such type
variable. That additional parameter is known as a \emph{dictionary}, it is a set
that will contain the type class function implementations for the type that
instantiates the type class constrained type variable. Call sites for those
function definitions are then augmented to pass the dictionary of the
appropriate type - the compiler decides what the appropriate type is based on
the parameter passed at the call site.

In the case of the \texttt{Show} type class above, we would expect the
\texttt{Bool} instance definition to be translated into the following
dictionary:
\begin{lstlisting}[mathescape]
    showBoolDict :: Bool -> String
    showBoolDict = show where
        show b = if b then "True" else "False"
\end{lstlisting}

\noindent The \texttt{print} function is then translated to the following:
\begin{lstlisting}[mathescape]
    print :: (a -> String) -> a -> IO ()
    print dict a = putStrLn (dict a)
\end{lstlisting}

\noindent Call sites \texttt{print b} where \texttt{b} is of type \texttt{Bool}
are then rewritten as \texttt{print showBoolDict b}.

\subsection{Dependent Object Types (DOT)}
DOT is a calculus intended as a step towards a theoretical foundation for the
programming language Scala and its type system \cite{AMO12}. DOT models a
restricted subset of Scala -- the base-calculus includes Scala's key features
from a type theory perspective, such as path-dependency, abstract type members
and a subtyping hierarchy with maximal type $\top$ and minimal type $\bot$.
Omitted are features of Scala that are less important from this perspective,
such as traits, classes and inheritance. We now look briefly at two important
features of DOT.

\paragraph{Abstract Type Members}
Abstract type members are a feature of Scala that allow for generic programming.
In Scala, a trait, class or object may declare an abstract type member. The
trait, class or object may then declare or define methods over that type. Below
is a trait \texttt{A} with an abstract type member \texttt{B}. The trait also
contains a method member \texttt{consumeB} that consumes a value of type
\texttt{B}.

\begin{minipage}{\linewidth}
\begin{lstlisting}[mathescape]
trait A {
    type B
    def consumeB(b: B): String
}
\end{lstlisting}
\end{minipage}

\noindent Any object may then declare itself a subtype of \texttt{A}, and
provide a concrete type in place of the abstract type member. Such an object may
implement method members over that type:

\begin{minipage}{\linewidth}
\begin{lstlisting}[mathescape]
object C extends A {
    type B = Int
    def consumeB(b: Int): String =
      b.toString
}
\end{lstlisting}
\end{minipage}

\noindent Like Scala, DOT also includes abstract type members, and thus the
above pattern is possible. We write $\TDEC{A}{S}{S'}$ to denote an object with a
type member $A$, whose lower bound is $S$ and upper bound is $S'$. Scala's
syntax \texttt{type T = U} is then equivalent to DOT's $\TDEC{T}{U}{U}$ - we let
\texttt{U} be the upper and lower bound. In DOT we can define a completely
abstract type member (equivalent to just \texttt{type T} in Scala) by using
$\bot$ and $\top$ as lower and upper bounds respectively. Conversely we can
define a fully specified type member by using a single specified type as both
the lower and upper bound.

\paragraph{Path-dependent Types}
Path-dependent types are a restricted form of dependent types \cite{AGORS16}.
Instead of allowing arbitrary computations over values in types, objects with
type members are the only values allowed, and selection of type members is the
only permitted operation on those objects.

Path-dependent types with function arrows can be used to recover
Hindley-Damas-Milner style polymorphism despite the absence of type variables,
via the passing of an object with a type member \cite{AGORS16}. Consider the
function \texttt{id} in an ML-like language with Hindley-Damas-Milner
polymorphism as below:

\begin{minipage}{\linewidth}
\begin{lstlisting}[mathescape]
    id : $\forall$a. a $\rightarrow$ a
    id x = x
\end{lstlisting}
\end{minipage}

We can rewrite this in DOT as below, with an additional parameter used to pass
the type of the parameter x:

\begin{minipage}{\linewidth}
\begin{lstlisting}[mathescape]
    id : $\forall$(u: {A: S}) $\forall$(v: u.A) u.A
    id = $\lambda$(u: {A: S}) $\lambda$(x: u.A) x
\end{lstlisting}
\end{minipage}

\subsection{Dependent object types with Implicit Functions (DIF)}
Implicits are widely used in Scala. The \emph{Dotty} Scala compiler \cite{O17},
written in Scala, contains at more than 5000 occurrences of the
\texttt{implicit} keyword. Akka \cite{H12}, a popular Scala library for
Actor-based concurrency, uses implicits at the core of its API. Therefore it is
of importance to the Scala community to have a solid theoretical foundation for
the correctness of Scala's implicit language constructs. Indeed the safety of
implicit functions in Scala has been evidenced by the type-safe integration of
implicit functions into lambda calculus \cite{OBLB18}. This evidence could be
further strengthened by their type-safe integration into DOT, the calculus on
which the Dotty compiler is based. In the remainder of this paper, we present
DIF, a type-safe integration of implicit functions into DOT. DIF is shown to be
type-safe by translation into the DOT calculus presented in \cite{AGORS16}. We
demonstrate that the type class pattern \cite{OMO10}, a typical use case of
implicit functions in Scala, can be translated typably into DIF, which recovers
ad-hoc polymorphism in DOT via the use of path-dependent types and abstract type
members to simulate Hindley-Damas-Milner parametric polymorphism.

\paragraph{Example} The type class pattern \cite{OBLB18} in DIF maps closely to
the type class pattern as is used in Scala. The example leverages DOT's
path-dependency, which exhibits parametric polymorphism by passing objects with
abstract type members. The dictionary passing of languages with type classes can
then be implemented with implicit functions. Example 1 shows the type class
pattern in Scala, and example 2 shows the type class pattern adapted for DIF.

\begin{figure*}[h]
\begin{lstlisting}[mathescape]
    trait Ord[T] {
        def compare(x: A, y: A): Boolean
    }

    def comp[T](x: A, y: A)(implicit ev: Ord[T]): Boolean = ev.compare(x, y)

    implicit def intOrd: Ord[Int] = new Ord[Int] {
        def compare(x: Int, y: Int): Boolean = a < b
    }
    ...
    comp(1, 2)
    ...
\end{lstlisting}
\caption*{\textbf{Example 1.} The type class pattern in Scala}
\end{figure*}

\begin{figure*}[h]
\begin{lstlisting}[mathescape]
    let ord_package = $\nu$(ord_p) {
        Ord = $\mu$(self: {
            A: $\bot$..$\top$
            compare: $\forall$(x: self.A, y: self.A)Boolean
        })
        comp: $\forall$(t: {A: $\bot$..$\top$})$\forall$(x: t.A, y: t.A)$\forall$$\IMP$(ev: ord_p.Ord$\wedge${A: $\bot$..$\top$})Boolean =
            $\lambda$(t: {A: $\bot$..$\top$})$\lambda$(x: t.A, y: t.A)$\IMP$.compare(x, y)
    } in
    let $\IMP$: ord_package.Ord$\wedge${A: Int..Int} = $\nu$(self: {
            A = Int
            compare: $\forall$(x: self.A, y: self.A)Boolean =
                $\lambda$(x: self.A, y: self.A) x < y
        })
    in
    ...
    ord_package.comp(1, 2)
    ...
\end{lstlisting}
\caption*{\textbf{Example 2.} The type class pattern in DIF}
\end{figure*}
