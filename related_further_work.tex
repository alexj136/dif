\section{Related and Further Work}

It is known that type assignment and subtyping are undecidable in DOT
\cite{AR16} since DOT can encode system $F_{<:}$. As DIF is a superset of DOT,
these properties hold for DIF transitively. DOT does, however, have a local type
inference procedure \cite{PT00}. It follows that such a procedure should also
exist for DIF, especially since local type inference is used to great effect in
Scala, a language with implicits. DOT has no principal types, and it is
therefore likely that DIF also lacks principal types, although this remains to
be shown.

It is well-known that there is a correspondence between implicits and type
classes - it is easily seen that the dictionary passing of type classes is
simulated with an implicit parameter. This has been observed in particular by
\cite{OMO10} and \cite{OSCLYW12}. It would be interesting to establish this
correspondence formally -- one can envision a translation $t_{c \rightarrow i}$
from type classes to implicits, and given the established results of type
classes to lambda calculus $t_{c \rightarrow l}$ \cite{WB89} and implicits to
lambda calculus $t_{i \rightarrow l}$ \cite{OBLB18}, it seems that it should be
possible to establish $t_{c \rightarrow i} \cdot t_{i \rightarrow l} \equiv t_{c
\rightarrow l}$ for some notion of $\equiv$.

It has not been investigated whether implicits can be used to encode type
classes. One can envison a translation $t_{i \rightarrow c}$ from
implicits to type classes, and it should then be expected that $t_{i
\rightarrow c} \cdot t_{c \rightarrow i} \equiv t_{c \rightarrow i} \cdot t_{i
\rightarrow c} \equiv \mathtt{id}$, again for some $\equiv$.

Implicit program constructs for session-typed concurrency have been studied in
\cite{JB18}. While Scala concurrency libraries leverage implicits, it is unclear
whether their usage is orthogonal to concurrency, or whether there are deeper
connections between the two concepts that can be further studied in theory. This
is a topic for further study.
