\begin{frame}
    \titlepage
\end{frame}

\begin{frame}
    \frametitle{Introduction}
    \begin{itemize}
        \pause
        \item Background
        \begin{itemize}
            \item DOT \hfill \textcolor{grey}{\emph{Amin, Moors, Odersky, 2012}}
            \item Implicits in Scala \hfill \textcolor{grey}{\emph{Odersky,
                Blanvillain et al, 2018}}
            \item Type classes \hfill \textcolor{grey}{\emph{Hall, Hammond et
                al, 1996}}
            \item Type classes via implicits \hfill
                \textcolor{grey}{\emph{Oliveira, Moors, Odersky, 2010}}
        \end{itemize}
        \pause
        \item Our calculus DIF that combines DOT with implicits - theory and
            applications
        \pause
        \item Related and further work
    \end{itemize}
\end{frame}

\begin{frame}
    \singletitle{Background}
\end{frame}

\begin{frame}
    \frametitle{Dependent Object Types (DOT)}
    \begin{itemize}
        \pause
        \item Foundational calculus for scala
        \begin{itemize}
            \pause
            \item Soundness result for (the core of) Scala's type system
        \end{itemize}
        \pause
        \item Very limited subset of Scala
        \begin{itemize}
            \pause
            \item Abstract type members, path dependent types
        \end{itemize}
        \pause
        \item Scope to expand DOT's coverage of Scala\pause, specifically...
    \end{itemize}
\end{frame}

\begin{frame}
    \frametitle{Implicits}
    \begin{itemize}
        \pause
        \item A mechanism for improving program readability
        \pause
        \item Generalisation of \textbf{default parameters}:
        \begin{itemize}
            \pause
            \item specify a default parameter that we can omit at call sites
            \pause
            \item \texttt{def f(x:Int, y:Float = 0.01) = ...}
            \pause
            \item \texttt{f(10)} becomes \texttt{f(10, 0.01)}
            \pause
            \item Problem: inflexible, not general
            \pause
            \item What if we want to vary the default for different parts of our
                codebase?
        \end{itemize}
        \pause
        \item Solution: \textbf{implicit parameters} - detach the declaration
            that a parameter is provided implicitly from the function itself
        \item Choose a default based on the call site's context
    \end{itemize}
\end{frame}

\begin{frame}[fragile]
    \frametitle{Implicits}
    \pause
    \begin{overprint}
        \onslide<2>
        \begin{lstlisting}[mathescape]
    def f(x: Int, implicit y: Float) = ...
    ...
    {
        implicit val v: Float = 4.0
        f(2)
    }
    ...
    {
        implicit val v: Float = 3.1415
        f(5)
    }
    ...
        \end{lstlisting}
        \onslide<3>
        \begin{lstlisting}[mathescape]
    def f(x: Int, implicit y: Float) = ...
    ...
    {
        implicit val v: Float = 4.0
        f(2, 4.0)
    }
    ...
    {
        implicit val v: Float = 3.1415
        f(5, 3.1415)
    }
    ...
        \end{lstlisting}
    \end{overprint}
\end{frame}

\begin{frame}[fragile]
    \frametitle{Implicits}
    \begin{itemize}
        \item Really useful for passing context \\
        \begin{overprint}
        \onslide<2>
        \begin{lstlisting}[mathescape]
    def a(i: A, z: Context): C = ...
    def b(j: B, z: Context): C = ...
    val ctx: Context = ...
    a(i1, ctx) + b(j1, ctx) + a(i2, ctx)
        \end{lstlisting}
        \onslide<3>
        \begin{lstlisting}[mathescape]
    def a(i: A, implicit z: Context): C = ...
    def b(j: B, implicit z: Context): C = ...
    implicit val ctx: Context = ...
    a(i1) + b(j1) + a(i2)
        \end{lstlisting}
        \end{overprint}
    \end{itemize}
\end{frame}

\begin{frame}[fragile]
    \frametitle{Type classes}
    \begin{itemize}
        \pause
        \item Ad-hoc polymorphism (\emph{overloading}) for Haskell \\
        \pause
        \begin{lstlisting}[mathescape]
    print :: Show a => a -> IO ()@\pause@
    class Show a where
        show :: a -> String@\pause@
    instance Show Int where
        show i = intToString i
    instance Show Bool where
        show b = if b then "True" else "False"@\pause@
    print a = putStrLn (show a)
        \end{lstlisting}
    \end{itemize}
\end{frame}

\begin{frame}[fragile]
    \frametitle{Type classes - implementation}
        \begin{lstlisting}[mathescape]
    type Show a = (a -> String)
    print :: Show a -> a -> IO ()@\pause@
    show 1
    show True@\pause@
    show intToString 1
    show (\b -> if b then "True"
                     else "False") True@\pause@
        \end{lstlisting}
\end{frame}

\begin{frame}[fragile]
    \frametitle{Type classes via implicits}
    \begin{itemize}
        \pause
        \item The parameter the compiler inserts is an implicit parameter!
        \pause
        \begin{lstlisting}[mathescape]
    implicit def intShow(x: Int): String =
        x.toString
    implicit def showBool(b: Bool): String =
        if(b) "true" else "false"
    def print[A](a: A, implicit show:
        A => String): Unit = println(show(a))
    print(1)
    print(true)
        \end{lstlisting}
    \end{itemize}
\end{frame}

\begin{frame}
    \frametitle{The calculus DIF}
    \begin{itemize}
        \pause \item Soundness via a translation from DIF programs to DOT
            programs
        \begin{itemize}
            \pause \item This follows the approach of the SI calculus, which
                gives a translation from lambda with implicit functions to
                System-F as a basis for Scala's implicits.
        \end{itemize}
        \pause \item A typable DIF program always translates to a typable DOT
            program
        \pause \item Type classes possible in DIF via implicits
    \end{itemize}
\end{frame}

\begin{frame}
    \frametitle{Ambiguity}
    \begin{itemize}
        \pause \item Scala: two criteria
        \begin{itemize}
            \pause \item More deeply nested variable: +1 point
            \pause \item More specific type: +1 point
            \pause \item If two variables tie, don't compile
        \end{itemize}
        \pause \item Haskell: reject all ambiguous programs
        \begin{itemize}
            \pause \item $\equiv$ one instance for each type class and type
        \end{itemize}
        \pause \item DIF: follow Scala's rules
        \begin{itemize}
            \pause \item No correctness proof for ambiguity resolution as
                Scala's solution is not formalised
        \end{itemize}
    \end{itemize}
\end{frame}

\begin{frame}
    \frametitle{Related work - coherence}
    \begin{itemize}
        \pause \item Calculi coherent when there is exactly one way of
            translating out implicits
        \pause \item COCHIS - theory of coherent implicits \\
            \textcolor{grey}{\emph{Schrijvers, Oliveira, Wadler, 2017}}
        \pause \item Coherence arguably not desirable in practice
        \pause \item Scala implicits are not coherent
        \pause \item DIF exists in the middle ground between theory (prefering
            coherence) and practice (prefering (limited) ambiguity)
    \end{itemize}
\end{frame}

\begin{frame}
    \frametitle{Further work}
    \begin{itemize}
        \pause \item Formal correspondence between type classes and implicits
        \begin{itemize}
            \pause \item Type classes a special case of implicits
        \end{itemize}
        \pause \item Type inference for DIF
        \begin{itemize}
            \pause \item Requires type inference for DOT
        \end{itemize}
    \end{itemize}
\end{frame}

\begin{frame}
    \singletitle{Thank you}
\end{frame}
